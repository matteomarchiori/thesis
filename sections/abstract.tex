\cleardoublepage
\phantomsection
\addcontentsline{toc}{chapter}{Abstract}

\begin{abstract}

    This thesis address the theme of load testing of cloud-native applications, and how it can be used to check their scalability. 
    The objective of this work was to develop resources that can be used to test cloud-native applications with different architectures, but common characteristics.

    A category of cloud-native application that everyday we use is the one of chat applications to send messages and to keep in contact with people.
    The most known and used around the world are WhatsApp, Telegram and Facebook Messenger \parencite{mehner_2021}.

    An interesting example of chat application is Signal \parencite{github}, because it is fully open source. This means that also the source code on the server side is available and deployable, which is not common among cloud-native applications.

    The architecture of Signal needs to be scalable in order to provide a good service to its users, but it cannot be taken for granted.

    What I do in the thesis is to design some load tests and use them on Signal.
    This is done to test its scalability in specific scenarios, which can be compared to real situations, and to have some tests that can be used on similar applications to get details about their scalability.

    The applications which I refer to are cloud-native applications, so applications which communicate to the world using the HTTP protocol.
    Signal is used because it is open source, so it provides a lot of information about its architecture, but the experiments are designed to be used also in other similar applications.

    The thesis is organized as follows: the first chapter (see chapter \vref{sec:problemstatement}) states the problem, so it goes more into the details of scalability, load testing, related works and the faced problem.
    The second chapter (see chapter \vref{sec:solutionspace}) describes the perimeter of intervention, where there is a discussion of the space explored with the experiments and what has not been covered.
    The third chapter (see chapter \vref{sec:experimentalresults}) discuss the experiments, the metrics and the obtained results.
    The last chapter (see chapter \vref{sec:conclusionsoutlook}) is about a conclusion to the problem and possible outlooks.
\end{abstract}
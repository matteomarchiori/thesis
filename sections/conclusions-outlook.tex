\chapter{Conclusions and outlook\label{sec:conclusionsoutlook}}

Da quanto confermano gli esperimenti\dots

Abbiamo visto come la scalabilità di Signal si comporta in queste situazioni...

Quindi in definitiva si può dire che\dots

Outlooks: NON SOLO SIGNAL. Dove posso usare la batteria di test creata? Per cosa?

Altre aree che possono essere prese in considerazione, in base anche alla letteratura.

*****************

The load tests are used to simulate situations which are closed to the reality, in order to understand how it behaves under an experimental load.

The problem is to understand how to implement a set of significant load tests, which should cover some real situations that the application can face.

To do this we designed the tests to simulate real situations but with scaled down numbers, reduced, because the used resources are a low number. This choice has been done to easily put under load the application.

The different kind of performed experiments can be compared to a normal use of Signal, so a low load during a long period of time, and huge load in a short period of time, which can be compared to a real situation such as the New Year's Eve wishes.

The lower load experiments with a higher duration refer to a single sequence of APIs calls to Signal, to complete an operation between an account creation and a message sending.
This operation gets completed by a single user, for $1000$ consecutive executions (see section \vref{sec:baseloadsettings}).

Instead, the higher load  experiments with a lower duration refer to a maximum of $100$ parallel users which can send single sequences of APIs calls, with a ramp-up period of $200$ seconds. This means that a thread (which correspond to a user) is launched every $2$ seconds. In these experiments the repetitions are $10$ instead of $1000$ (see section \vref{sec:highloadsettings}).

As the results show both of the architectures are able to face the different loads, with some differences among the experiments.

Resources that are needed to afford huge traffic cost money. This is the reason why elastic scalability systems are needed to acquire and release resources on the right time, to reduce the waste of money in the best possible manner.

The experiments do not investigate on it. To do this, it is necessary to use a IAAS such as AWS is configured in a way similar to the production environment.
This is not under public knowledge.



What the experiments performed in the thesis confirmed is that the architectures used for the Signal server behave better in specific conditions, the 4.97 version is more suited for long term low loads (see figures \vref{fig:signalbasecreateloadold} and \vref{fig:signalbasemessageloadold} compared to figures \vref{fig:signalbasecreateloadnew} and \vref{fig:signalbasemessageloadnew}), while the 6.13 version supports better higher loads in short times (see figures \vref{fig:signalcreateloadnew} and \vref{fig:signalhighmessageloadnew} compared to figures \vref{fig:signalcreateloadold} and \vref{fig:signalmessageloadold}).
Moreover, the new version delegate the management of components to AWS, which is good to waste less time into them. 

None of the two architecture seem to have weak points directly connected to a specific API call.

What we demonstrate is that the evolution of the architecture with a change on its components is good to avoid huge loads in short time periods, but it can be less suited for long term loads.
Moreover, changes apported are useful to delegate the management of the database system and other services used by the server, so their performance must respect the ranges offered by AWS.

\section{Outlook\label{sec:outlook}}

It is possible to cover also the entire architecture of Signal to test if the AWS provided services respect the agreed level of resources.

Two examples are:
\begin{itemize}
    \item the voice/video call related components;
    \item the attachment related components.
\end{itemize}

Another outlook is related to the kind of experiments which can be performed.

For example, it is possible to load test the architecture in a white box way, so repeat the previous tests while Redis is running other operations, or while PostgreSQL is serving another application.

It would be possible to change the characteristics of the environment where the Signal server is hosted, to check how the results change.



Other possible outlooks are the following:
\begin{itemize}
    \item experimental analysis with more focus on each single component took by itself, with JMeter and plugins or other tools dedicated to this aim \cite{dubey_2019};
    \item analysis of the remaining API calls, as stated before;
    \item experiments on the AWS components, for example by using Distributed Load Testing \cite{king_dobner_2019};
    \item more realistic application scenarios based on real data, which can be provided by asking Open Whisper Systems \cite{wikipedia_2021}.
\end{itemize}
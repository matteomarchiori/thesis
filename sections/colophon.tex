%\hfill
%\vfill
%\noindent Matteo Marchiori: \textit{Migrazione di una Web application con architettura monolitica verso un'architettura a microservizi,} Tesi di laurea triennale, \textrm{\textcopyright}~26 settembre 2019.\\\\
%
%\noindent Per le immagini riportate in questa tesi :\\
%
%\noindent Ballet: si veda la figura~\vref{fig:soavsmicro2}\\
%\noindent Fonte: \url{https://pixabay.com/it/photos/balletto-teatro-danza-illuminazione-2682291}\\
%\noindent Foto originale di \href{https://pixabay.com/it/users/riomarbruno-3059709/}{Bruno Riomar}, 2017\\
%\noindent \textrm{\textcopyright}~\url{https://pixabay.com/it/photos/balletto-teatro-danza-illuminazione-2682291}\\
%\noindent Licenza: \url{https://pixabay.com/it/service/license}\\
%
%\noindent Orchestra: si veda la figura~\vref{fig:soavsmicro2}\\
%\noindent Fonte: \url{https://upload.wikimedia.org/wikipedia/commons/3/39/MITO_Orchestra_Sinfonica_RAI.jpg}\\
%\noindent Foto originale di \href{https://www.flickr.com/photos/28437914@N03}{MITO SettembreMusica}, 2008\\
%\noindent \textrm{\textcopyright}~\url{https://upload.wikimedia.org/wikipedia/commons/3/39/MITO_Orchestra_Sinfonica_RAI.jpg}\\
%\noindent Licenza: Creative Commons\\
%
%\noindent Circuit breaker: si veda la figura~\vref{fig:circuitbreaker}\\
%\noindent Fonte: \url{https://upload.wikimedia.org/wikipedia/commons/6/6a/Circuit_breaker_2_pole_on_DIN_rail.JPG}\\
%\noindent Foto originale di \href{https://commons.wikimedia.org/wiki/User:Kae}{Alexey Khrulev}, 2008\\
%\noindent \textrm{\textcopyright}~\url{https://upload.wikimedia.org/wikipedia/commons/6/6a/Circuit_breaker_2_pole_on_DIN_rail.JPG}\\
%\noindent Licenza: Creative Commons\\\\
%
%\noindent Tutti i marchi riportati appartengono ai legittimi proprietari.